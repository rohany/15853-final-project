\documentclass{article}

\author{
  Rohan Yadav\\
  \and
  Rahul Jaisingh\\
}
\title{15853 Final Project Report}
\date{}

\usepackage{listings}

\begin{document}
\maketitle

\section{Introduction}

The RAM model of computation is the model where
a processor can access any cell in memory in constant time.
%
This model is the standard setting for algorithm design, and
cost analysis under this model generally translates well to
runtime on moden machines.
%
However, as the data we run our algorithms on is quickly
becoming huge, the standard RAM model of analysis diverges from
reality.
%
Modern machines have non uniform memory access times,
in both onboard memories (cache heirarchies) and external memory
(RAM vs disk).
%
On these machines, accessing data that is not currently
in cache, or in RAM can be order of magnitudes slower than accessing
data that does reside in cache or RAM.
%
As data becomes much larger than the size of the cache, or the size of
RAM, these slowdowns can dominate the running time.
%
To better analyze the running time of algorithms on these data sets,
as well as to design algorithms to take advantage of this
difference in memory access times, we analyze programs in the I/O model.

In the I/O model, we assume that our input has size $N$, and there exists
the fast memory, which has size $M$, and is chunked into pieces of size $B$.
%
We don't assume anything about the size of the slow memory.
%
We consider all operations on data currently residing in $M$ to have cost 0.
%
The only operations that have a cost are memory loads in $M$ from the slow memory.
%
Data is loaded in $M$ from the slow memory in blocks of size $B$.
%
We assume that the algorithm can choose which data to pull into $M$,
and which blocks get evicted as well.
%
This model more accurately captures the effect of different memory access times
on modern machines, especially the discrepacny between RAM and disk access.
%
Additionally, the I/O model is a cache-aware model, which is different from
the cache-oblivious model, because in the I/O model we can assume that the
algorithm has access to the parameters $M$ and $B$, where in the cache-oblivious
model the algorithm does not.
%


In this report, we implemented and experimented with two I/O efficient
sorting algorithms, and ran them on data that did not fit in RAM.

\section{Sorting}
Optimal sorting algorithms in the RAM model translate poorly to the
I/O model, as they generally perform too much random access into the data,
or do not take advantage of the fast access of data residing in fast memory.
%
The lower bound for sorting in the I/O model is $\Theta((N/B)\log_{M/B}(N/M))$ time,
which is much faster than the lower bound for sorting in the RAM model.
%
This bound is proven in lecture, and is a similar argument to the standard
information theoretic argument for the RAM model lower bound.
%
Standard sorting algorithms like Mergesort or Quicksort are able to achieve
only $\Theta((N/B)\log_{2}(N/M))$ time in the I/O model, which is a large factor
slower than the lower bound.
%
To achieve the optimal bounds for the I/O model, we implement a K-way
mergesort algorithm, as well as a buffer tree based sort.

\section{K-way Mergesort}

\subsection{Introduction}
The basic idea of the K-way Mergesort algorithm is to break
the input sequence into more than two peices in
the recursive calls.
%
Doing this allows us to have the base of the logarithm be larger
than 2.
%
Specifically, we break our input sequence into $M/B$ peices,
and perform a recursive I/O sort on each one.
%
This allows the depth of the recursion to be $\Theta(\log_{M/B}(N/M))$,
because we recurse until the data is size $M$.
%
The merging of all $M/B$ sequences is also efficient, due to the number
of sequences we are merging together.
%
Because we can load $B$ elements into $M$ for each sequence, and
we only need to read any element from each sequence once, we can
have a block from each sequence in $M$ at once, and have
fast access to a portion of every sequence that is being merged.
%
Therefore, this entire merge step costs $N/B$, since every element must be
loaded, and all other operations are free in memory.
%
The recurrence generated by this algorithm is balanced, and results in
the final bounds of $\Theta((N/B)\log_{M/B}(N/M))$.

\subsection{Algorithm and Optimizations}
More detailed psuedocode for our algorithm can be found in Figure~\ref{k-merge}.
%
We made a few optimizations while implementing this algorithm.
%
The first was to use a priority queue to hold an element from
each of the $M/B$ sequences, rather than looping over
all of them to find the minimum every time.
%
While theoretically this operation is free, in practice $M/B$ can be large,
and accessing all of these elements can be costly.
%
One detail that we couldn't avoid was that we had to allocate extra output
arrays, because the multi-way merge couldn't be done in place.
%
This extra allocation in a sense pulls extra memory into $M$ and forces extra
blocks to be evicted from $M$ during the merges.
%
While this only affects the number of loads by a constant factor,
there is slowdown during the runtime due to the extra memory usage.

\begin{figure*}
\begin{lstlisting}[mathescape=true]
  fun kmergesort $S$ =
    (* base case in memory *)
    if $|S| <= M$ then return quicksort $S$ else

    (* recursive case *)
    $P$ = split into $M / B$ pieces
    for $p$ in $P$:
      kmergesort $p$

    $O$ = allocate output sequence
    (* current index at each piece *)
    $F$ = {0 for $0 \leq i < M/B$}

    for i in range(0, $|S|$):
      $(m, j)$ = min element from $P$ at each index
      $O$[i] = m
      $F$[j]++

    return $O$
\end{lstlisting}
\caption{K-way Mergesort Psuedocode}
\label{k-merge}
\end{figure*}


\section{Buffer Tree Sort}

\subsection{Introduction}

B-trees are I/O efficient data structures that provide good
bounds on searching in the I/O model, and used heavily in practice.
%
B-trees are trees that have a higher fanout that two,
and also store more than a single element at each node.
%
In the case of I/O efficient B-trees, the trees store
$B$ elements at each node, and have $B$ children.
%
This leads to fast lookups, as searching for an element
takes $\log_B N$ time, and the search through every
node takes only constant time, as the block has already
been loaded into memory.
%
However, if a standard B-tree is directly used in
sorting, sub-optimal bounds are achieved --- $\Theta(N\log_B(N))$.
%
The reason for this is that insertions into the tree do not take
advantage of the caching in the I/O model.


To achieve the optimal bounds, we use a data structure called
a Buffer Tree, or a Buffered B-tree.
%
A Buffer Tree is similar to a B-tree, but every internal node
holds a buffer of size $M$, and every leaf node has a buffer
of size $B$.
%
Additionally, every internal node has $M/B$ children.
%
Upon an insertion, an element is just added to the appropriate buffer,
as opposed to moving down the tree.
%
Once a buffer fills up, all of the elements are inserted into the appropriate
children.
%
In this way, operations are batched to be more efficient, and now a sequence
of $N$ insertions costs $\Theta(N/B \log_{M/B}(N/B))$.
%
To recover a sorted sequence, we can simply walk across the leaves of the tree.
%
Because this final step takes only $N/B$ time, we can use a Buffer Tree
to construct an I/O optimal sorting algorithm.

\subsection{Algorithm and Implementation}
The main difficulty in our implementation was the insertion function
for our buffer trees.
%
Searching and writing out the sorted values to the output sequence
were straightforward.
%
The difficulty in insertion was mainly the internal balancing steps.
%
Psueodocode for our insertion algorithm can be found in Figure~\ref{buf-insert}.
%
To effectively sort, we also needed to add in a flush style operation.
%
The main issue without an operation like this is that the buffers themselves
are not sorted, and elements can be stuck in the buffers instead of being sorted
at the leaves, by the time that all $N$ elements have been inserted.
%
To recover a sorted sequence, we need to flush all of the buffers to
make sure that all of the elements that in the leaves, so that we can just walk
over the leaves to output the sorted elements.
%
To do the flush, we just sort the current buffer, and then recursively insert each
element into its corresponding child, and then flush each child.
%
After this operation, all of the buffers will be flushed, and all of the elements will
be sorted at the leaves.




\begin{figure*}

  \begin{lstlisting}[mathescape=true]
    fun rec_insert$(T, e)$ =
      insert $e$ into $T$->buffer
      if $T$ is a leaf: return
      if $T->$buffer is full:
        sort$(T$->buffer$)$
        for each $d$ in $T$->buffer:
          $c$ = child of $T$ that $d$ belongs in
          rec_insert$(c, d)$
          if $c$ is too large:
            split$(c)$ into $c_1$ and $c_2$
            insert $c_1$ and $c_2$ into $T$'s children

    fun insert$(T, e)$ =
      $T'$ = rec_insert$(T$->root, $e)$
      if $T'$->root is too big:
        $R$ = new root
        $R$->children = split$(T)$
        return $R$
      return $T'$
  \end{lstlisting}
  \caption{Buffer Tree insertion algorithm}
  \label{buf-insert}
\end{figure*}

\section{Results}

\subsection{Experimental Details}
We implemented our algorithms in C++, and compiled with \texttt{O3} level
optimization.
%
We compared against the sorting algorithm provided by the
\texttt{<algorithm>} library in the STL.
%
We additionally used the same sorting algorithm as the base case for our
sorting algorithms when the data fit into memory.
%
We tested on an AWS EC2 instance with a single core Intel Xeon CPU @3.30 GHz and 0.5 GB of RAM.
%
Additionally, we outfitted the instance with a 50 GB swap file, so allocations
larger than RAM are allocated onto the swap file, which resides in external memory.
%
Allowing the instance to have only 0.5 GB of RAM allowed for easier testing,
as we were able to allocate much smaller data and have it still spill out of
on board memory.

\subsection{K-way Mergesort Results}

\begin{figure}
  \centering
  \begin{tabular}{|c|c|c|}
    \hline
    Input Size & Standard Time (Seconds) & I/O Time (Seconds)\\
    \hline
    \hline
    2 GB & 6309 & 3919 \\
    \hline
    4 GB & 190721 & 78673 \\
    \hline
    8 GB & 184297 & 148170 \\
    \hline
  \end{tabular}
  \caption{Results of K-way Mergesort}
  \label{k-merge-res}
\end{figure}

The results of our k-way mergesort algorithm can be found in Figure~\ref{k-merge-res}.
%
We ran these experiments with $M = 0.5$GB and $B = 16$ KB.
%
While the page size on the instance was $4$ KB, we saw better results with a larger $B$
value.
%
We postulate that this was the case due to the fewer number of recursive calls that must be
made by the sort, which could incur a large amount of overhead, as $M/B$ is large in these cases.
%
We can see that our implementation performs well, consistently out-performing the sort
provided in the STL, and even doubling the performance on the 2 and 4 GB input sizes.

\subsection{Buffer Tree Sort}

\begin{figure}
  \centering
  \begin{tabular}{|c|c|c|}
    \hline
    Input Size & Standard Time (Seconds) & I/O Time (Seconds)\\
    \hline
    \hline
    wei & wei & wei \\
    \hline
  \end{tabular}
  \caption{Results of Buffer Tree sort}
  \label{buffer-tree-res}
\end{figure}

The results of our buffer tree sort can be found in Figure~\ref{buffer-tree-res}.



\section{Conclusions}

In conclusion, we can see that for TODO: at least k-merge,
algorithms that are optimal in the standard model do not accurately
account for non uniform memory access costs.
%
By analyzing algorithms even in the I/O model, which assumes
a simple memory hierarchy, and user control over memory,
we can write algorithms that heavily outperform
even the standard libraries of C++.

\section{References}

For all references we consulted the course notes of 15853.
%
Additionally, for some more details on buffer trees we looked at the
original paper by Lars Arge.

\end{document}

